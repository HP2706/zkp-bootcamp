\documentclass{article}
\usepackage{amsmath}
\usepackage{amssymb}

\begin{document}

\textbf{Problem set (mandatory):}

\begin{enumerate}
    \item Suppose you have a set $\{\text{monyet}, \text{kodok}, \text{burung}, \text{ular}\}$. Define a binary operator that turns it into a group using set-theoretic definitions.
    
    \begin{enumerate}
        \item the group has an identity element
        \item for any element $a$ in the group there exists an inverse element $a^{-1}$ such that $a * a^{-1} = a^{-1} * a = e$
        \item the binary operator is associative ie $(a * b) * c = a * (b * c)$
        \item the group is closed under the binary operator ie $a * b$ is in the group
    \end{enumerate}

    to find such an operator we first need to define an identity element. let that be "monyet".
    TODO

    \item Find a binary operator that is closed but not associative for real numbers.
    exponentiation ie for $a, b \in \mathbb{R}: a^b \in \mathbb{R} $ is not associative for real numbers.
    
    \item Let our set be real numbers. Show a binary operator that is not closed.
    the binary operator $\sqrt{a}$ is not closed because for $a \in -\mathbb{R}$ 
    
    \item What algebraic structure is all odd integers under multiplication? All even integers under addition?
    \begin{enumerate}
        \item odd integers under multiplication is not closed, associative, has identity element 1.
        \item a semigroup is a set with a binary operator that is associative and closed(you cannot get an uneven number from the addition of two even numbers)
    \end{enumerate}

    \item Let our group be $3 \times 2$ matrices of integers under addition. What is the identity and inverse? Can this be a cyclic group, why or why not? (Pay very close attention to the definition of cyclic group)
    \begin{enumerate}
        \item The identity is the zero matrix of shape 3x2
        \item The inverse of any matrix A is -A
        \item This cannot be a cyclic group because it is not generated by a single element. The group is infinite and has multiple generators.
    \end{enumerate}
    \item Demonstrate that
    \[
    n \pmod p, n=...-2,-1,0,1,2,...
    \]
    is a group under addition. Remember, you need to show that:
    \begin{itemize}
        \item the binary operator is closed
        \item the binary operator is associative
        \item an identity exists
        \item every element has an inverse
    \end{itemize}
    
    \begin{enumerate}
        \item Closure: For any $a, b \in \mathbb{Z}/p\mathbb{Z}$, $(a + b) \bmod p$ is also in $\mathbb{Z}/p\mathbb{Z}$. 
           This is because the result of $(a + b) \bmod p$ will always be in the range $[0, p-1]$, which are precisely the elements of our group.
        \item The binary operator is associative as addition is associative 
        \item The identity is 0
        \item The inverse of an element $a$ is $-a \bmod p$ or equivalently $(p-a) \bmod p$
    \end{enumerate}


    \item Demonstrate that
    \[
    g^{n} \pmod p, n=...-2, -1, 0, 1, 2...
    \]
    Where $g$ and $p$ are relatively prime is a group under multiplication. That is, given elements $g^a$, $g^b$, $(g^a)*(g^b)$ is in the group and the binary operator follows the group laws.
    
    we need to show
    \begin{enumerate}
        \item the binary operator is closed:
        Let $g^a \bmod p$ and $g^b \bmod p$ be any two elements in the group.
        Their product is $(g^a \bmod p) \cdot (g^b \bmod p) \bmod p = g^{a+b} \bmod p$.
        Since $g^{a+b} \bmod p$ is of the form $g^n \bmod p$ for some integer $n$, it is also an element of the group.
        Therefore, the binary operator is closed.
        
        \item the binary operator is associative as $g^{a+b} = g^a * g^b$
        \item an identity exists as $g^0 = 1$
        \item every element has an inverse as $g^{-a} = \frac{1}{g^a}$
    \end{enumerate}


    \item Both integers and polynomials with integer coefficients are rings. It is possible to define a homomorphism from integers to polynomials and polynomials to integers, but it isn't the same transformation.

    
\end{enumerate}

\end{document}