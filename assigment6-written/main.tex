\documentclass{article}
\usepackage{hyperref}
\usepackage{amsmath}
\usepackage{amssymb}
\usepackage{bookmark}
\begin{document}
\section{Practice Problems}

\begin{enumerate}
    \item Create an arithmetic circuit that takes signals $x_1$, $x_2$, \ldots, $x_n$ and is satisfied if \emph{at least} one signal is 0.

     we note that this is effectively the same as saying $\sum_{i=1}^n x_i === n-1$ 

    \item Create an arithmetic circuit that takes signals $x_1$, $x_2$, \ldots, $x_n$ and is satisfied if all signals are 1.

     \begin{align*}
          \sum_{i=1}^n x_i - n &= 0 \\
          \text{and}
          \forall i \in [0, n]: x_i(1-x_i) == 0
     \end{align*}

    \item A bipartite graph is a graph that can be colored with two colors such that no two neighboring nodes share the same color. Devise an arithmetic circuit scheme to show you have a valid witness of a 2-coloring of a graph. Hint: the scheme in this tutorial needs to be adjusted before it will work with a 2-coloring.
    TODO
    \item Create an arithmetic circuit that constrains $k$ to be the maximum of $x$, $y$, or $z$. That is, $k$ should be equal to $x$ if $x$ is the maximum value, and same for $y$ and $z$.
    
     for two variables we can simulate the max function like so
     \begin{align*}
          max(x,y) &= \frac{x+y + |x-y|}{2}
     \end{align*}
     we can extend this to three variables by using the fact that $max(x, max(y,z)) = max(max(x,y), z)$
     \begin{align*}
          max(x, max(y,z)) &= max(max(x,y), z)
     \end{align*}
     IE we need to satisfy

     \begin{align*}
          a &= \frac{x+y+|x-y|}{2} \\
          k &= \frac{a+z+|a-z|}{2}
     \end{align*}
    
    \item Create an arithmetic circuit that takes signals $x_1$, $x_2$, \ldots, $x_n$, constrains them to be binary, and outputs 1 if \emph{at least} one of the signals is 1. Hint: this is trickier than it looks. Consider combining what you learned in the first two problems and using the NOT gate.
          
     we note this is the same as saying \[\sum_{i=0}^{n} x_i \geq 1\]

     we can use a combination of the not operator and $\sum_{i=1}^n x_i === n-1$ 


     \begin{align*}
          \forall i \in [0, n]: x_i(1-x_i) === 0 \\
          a = \sum_{i=1}^n x_i === 0 \\
          res = 1 - a
     \end{align*}

    
    \item Create an arithmetic circuit to determine if a signal $v$ is a power of two (1, 2, 4, 8, etc). Hint: create an arithmetic circuit that constrains another set of signals to encode the binary representation of $v$, then place additional restrictions on those signals.
    
    first we constrain v 
    \begin{align*}
          v = \sum_{i=0}^{n} 2^i x_i \\
          \forall i \in [0, n]: x_i(1-x_i) === 0 \\
    \end{align*}

    we need an additional constraint that checks that exactly one bit can be set to one
    \begin{align*}
          \sum_{i=0}^{n} x_i = 1
    \end{align*}


    \item Create an arithmetic circuit that models the 
    Subset sum problem \href{https://en.wikipedia.org/wiki/Subset_sum_problem}{(link)}. Given a set of integers (assume they are all non-negative), determine if there is a subset that sums to a given value $k$. For example, given the set $\{3,5,17,21\}$ and $k = 22$, there is a subset $\{5, 17\}$ that sums to $22$. Of course, a subset sum problem does not necessarily have a solution.
    Use a "switch" that is 0 or 1 if a number is part of the subset or not.

    we define a set $A = \{a_1, a_2, \ldots, a_n\ : a \geq 0\}$ and a target value $k$

    we create a switch $a_i$ for each set-element indicating if it is part of the subset.
    \begin{align*}
          \sum_{i=1}^n a_i x_i === k \\
          \forall i \in [1, n]: x_i(1-x_i) === 0
     \end{align*}

    \item The covering set problem starts with a set $S = \{1, 2, \ldots, 10\}$ and several well-defined subsets of $S$, for example: $\{1, 2, 3\}$, $\{3, 5, 7, 9\}$, $\{8, 10\}$, $\{5, 6, 7, 8\}$, $\{2, 4, 6, 8\}$, and asks if we can take at most $k$ subsets of $S$ such that their union is $S$. In the example problem above, the answer for $k = 4$ is true because we can use $\{1, 2, 3\}$, $\{3, 5, 7, 9\}$, $\{8, 10\}$, $\{2, 4, 6, 8\}$. Note that for each problem, the subsets we can work with are determined at the beginning. We cannot construct the subsets ourselves. If we had been given the subsets $\{1,2,3\}$, $\{4,5\}$ $\{7,8,9,10\}$ then there would be no solution because the number $6$ is not in the subsets.
     On the other hand, if we had been given $S = \{1,2,3,4,5\}$ and the subsets $\{1\}$, $\{1,2\}$, $\{3, 4\}$, $\{1, 4, 5\}$ and asked can it be covered with $k = 2$ subsets, then there would be no solution. However, if $k = 3$ then a valid solution would be $\{1, 2\}$, $\{3, 4\}$, $\{1, 4, 5\}$.
     Our goal is to prove for a given set $S$ and a defined list of subsets of $S$, if we can pick a set of subsets such that their union is $S$. Specifically, the question is if we can do it with $k$ or fewer subsets. We wish to prove we know which $k$ (or fewer) subsets to use by encoding the problem as an arithmetic circuit.

     we define a universe $S = \{a_1, a_2, \ldots, a_10\}$\\
     and N subsets $S_i \subseteq S \forall i \in [1, N]$
     
     first we can constrain the problem by having a set $s$ which determines if $S_i$  is part of the minimal union solution.

     \begin{align*}
          \forall i \in [1, N]: s_i(1-s_i) === 0 \\
          \sum_{i=1}^N s_i === k \\
     \end{align*}

     we now need to find a way to encode union of subsets and avoid duplicates.
     we can do this by creating additional switches $s_{ij}$ for the jth element in the ith subset $a_{ij}$ specifying if the element is unique in the union.
     \begin{align*}
          \text{constrain each switch to be binary}
          \forall i \in [1, N]: \forall j \in [1, |S_i|]: s_{ij}(s_{ij}-1) === 0 \\
          \text{constrain each element to only be used once}
          \forall j \in [1, 10]: \sum_{i}^{N} s_{ij} === 1 \\
          \text{constrain the union of all subsets to be the universe}
          \sum_{i}^{N} \sum_{j}^{|S_i|} s_{ij}*a_ij === \sum_{0}^{10} a_i
     \end{align*}

     this proves that we know a k and subsets that cover the universe but not that we have the minimal k

     we can show we have the minimal k by summing over k-1 subsets and proving that the sum is less than the universe

     \begin{align*}
          A = \sum_{i}^{k-1} s_{ij}*a_ij === \sum_{0}^{10} a_i
          \text{apply the not gate}
          1 = 1 - A
     \end{align*}





\end{enumerate}

\end{document}