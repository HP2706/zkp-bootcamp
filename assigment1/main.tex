\documentclass{article}
\usepackage{amsmath}
\usepackage{hyperref}
\usepackage{amsmath}
\usepackage{listings}
\usepackage{xcolor}

\lstset{
    language=Python,
    basicstyle=\ttfamily\small,
    keywordstyle=\color{blue},
    stringstyle=\color{red},
    commentstyle=\color{green!60!black},
    numbers=left,
    numberstyle=\tiny,
    numbersep=5pt,
    backgroundcolor=\color{gray!10},
    frame=single,
    breaklines=true,
    breakatwhitespace=true,
    showstringspaces=false,
}
\begin{document}

\section*{Homework 1}

For all problems below, assume the finite field is \( p = 71 \).

\textbf{Remember, this is done in a finite field so your answer should only contain numbers [0-70] inclusive. There should be no fractions or negative numbers.}

\subsection*{Problem 1}
Find the elements in a finite field that are congruent to the following values:
\begin{itemize}
    \item -1 
    \item -4 
    \item -160
    \item 500
\end{itemize}

Solution: 

\begin{itemize}
    \item $-1 \equiv 70$ as $1 + 70 \mod 71 = 0$
    \item $-4 \equiv 67$ as $4 + 67 \mod 71 = 0$
    \item $-160 \equiv 53$ as $160 + 53 \mod 71 = 0$
    \item $500 \equiv 3$ as $500 \mod 71 = 3$
\end{itemize}

\subsection*{Problem 2}
Find the elements that are congruent to \( a = \frac{5}{6}, b = \frac{11}{12}, \) and \( c = \frac{21}{12} \).
Verify your answer by checking that \( a + b = c \) (in the finite field).

\begin{itemize}
    \item 1.
        we factorize and compute 5 * 1/6 seperately:
        $\frac{1}{6} * 6 = 1$ thus we want to find a number $x$ st
        $x*6 \equiv 1 \mod 71 \implies$ $x*6 = 72$  
        $x = 72/6 = 12$
        
        thus the answer is: $5 * 12 = 60$
    \item 2.
        we factorize and compute 11 * 1/12 seperately:
        $\frac{1}{12} * 12 = 1$ thus we want to find a number $x$ st
        $x*12 \equiv 1 \mod 71 \implies$ $x*12 = 72$
        $x = 72/12 = 6$
        thus the answer is: $11 * 6 = 66$
    \item 3.
        we factorize and compute 21 * 1/12 seperately:
        $\frac{1}{12} * 12 = 1$ thus we want to find a number $x$ st
        $x*12 \equiv 1 \mod 71 \implies$ $x*12 = 72$
        $x = 72/12 = 6$
        thus the answer is: $(21 * 6) \mod 71 = 126 \mod 71 = 55$
    \end{itemize}

    we can verify this by checking that $a + b = c$
    $60 + 66 = 126 \mod 71 = 55$

\subsection*{Problem 3}
Find the elements that are congruent to \( a = \frac{2}{3}, b = \frac{1}{2}, \) and \( c = \frac{1}{3} \).

\begin{itemize}
    \item a.
    we find $\frac{1}{3}$ by solving
    \[
        3x = 1 = 72 \mod 71 \implies x = \frac{72}{3} = 24
    \]
    thus the answer is $2 * 24 = 48$
    $\frac{2}{3} \equiv 48$

    \item b.
    we find $\frac{1}{2}$ by solving
    \[
        2x = 1 = 72 \mod 71 \implies x = \frac{72}{2} = 36
    \]
    $\frac{1}{2} \equiv 36$

    \item c.
    we find $\frac{1}{3}$ by solving
    \[
        3x = 1 = 72 \mod 71 \implies x = \frac{72}{3} = 24
    \]
    
    $\frac{1}{3} \equiv 24$

\end{itemize}

\[
(36 \times 48) \mod 71 = 24
\]

\subsection*{Problem 4}
What is the modular square root of 12?
Verify your answer by checking that \( x \cdot x = 12 \ (\text{mod}\ 71) \).

answer: 12
as
\[15^2 \mod 71 = 225 \mod 71 = 12\]

\subsection*{Problem 5}
The inverse of a \( 2 \times 2 \) matrix \( A \) is
\[
A^{-1} = \frac{1}{\det} \begin{bmatrix} d & -b \\ -c & a \end{bmatrix}
\]
where \( A \) is
\[
A = \begin{bmatrix} a & b \\ c & d \end{bmatrix}
\]
And the determinant \(\det\) is
\[
\det = a \times d - b \times c
\]

Compute the inverse of the following matrix:
\[
\begin{bmatrix} 1 & 1 \\ 1 & 4\end{bmatrix}
\]

Verify your answer by checking that
\[
A A^{-1} = I
\]
Where \( I \) is the identity matrix.


\[
det = 1 * 4 - 1 * 1 = 3
\]


\[
\frac{1}{3} \begin{bmatrix} 4 & -1 \\ -1 & 1 \end{bmatrix}
\]

we note that $\frac{1}{3} \equiv 24$ and $-1 \equiv 70$

thus we have the matrix 

\[
A^{-1} = 24 \begin{bmatrix} 4 & 70 \\ 70 & 1 \end{bmatrix} = \begin{bmatrix} 25 & 47 \\ 47 & 24 \end{bmatrix}
\]

\[
A A^{-1} = \begin{bmatrix} 1 & 1 \\ 1 & 4 \end{bmatrix} \begin{bmatrix} 25 & 47 \\ 47 & 24 \end{bmatrix} =\\
\begin{bmatrix}
(25 + 47) \mod 71  & (47 + 24) \mod 71 \\
(25 + 188) \mod 71 & (47 + 96) \mod 71     
\end{bmatrix} = \begin{bmatrix}
1 & 0 \\
0 & 1
\end{bmatrix}
\] 

\subsection*{Problem 6}
Suppose we have the following polynomials:
\[
p(x) = 52x^2 + 24x + 61
\]
\[
q(x) = 40x^2 + 40x + 58
\]

What is $p(x) + q(x)$? What is $p(x) \cdot q(x)$?

\[
p(x)+ q(x) = ((52+40) \mod 71)x^2 + ((24+40) \mod 71)x + ((61+58) \mod 71) = 21x^2 + 64x + 48
\]

\begin{equation}
    p(x) \cdot q(x) = (52x^2 + 24x + 61) \cdot (40x^2 + 40x + 58) = \\
    (52 \cdot 40) \cdot x^4 + (52 \cdot 40 + 24 \cdot 40) \cdot x^3 + 
    (52 \cdot 58 + 24 \cdot 40 + 61 \cdot 40) \cdot x^2 + (24 \cdot 58 + 61 \cdot 40) \cdot x + 61 \cdot 58 =
    21 \cdot x^4 + 58 \cdot x^3 + 26 \cdot x^2 + 69 \cdot x + 59
\end{equation}

Use the \texttt{galois} library in Python to find the roots of $p(x)$ and $q(x)$.


\begin{lstlisting}
    import galois
    from galois import Poly, GF
    
    # Define the finite field
    Field = GF(71)
    
    p_coeffs = Field([52, 24, 61])  # 52x^2 + 24x + 61
    q_coeffs = Field([40, 40, 58])  # 40x^2 + 40x + 58
    
    p = Poly(p_coeffs)
    q = Poly(q_coeffs)
    
    p_roots = p.roots()
    q_roots = q.roots()
    
    print("Roots of p(x):", p_roots)
    print("Roots of q(x):", q_roots)
    
    pq = p * q
    
    pq_roots = pq.roots()
    
    print("Roots of p(x) * q(x):", pq_roots)
\end{lstlisting}

The roots of $p(x)$, $q(x)$, and $p(x)q(x)$ are:
\begin{itemize}
    \item Roots of $p(x)$: \lstinline{p_roots}
    \item Roots of $q(x)$: \lstinline{q_roots}
    \item Roots of $p(x)q(x)$: \lstinline{pq_roots}
\end{itemize}

What are the roots of $p(x)q(x)$?

\subsection*{Problem 7}
Find a polynomial $f(x)$ that crosses the points $(10, 15)$, $(23, 29)$

Since these are two points, the polynomial will be of degree 1 and be the equation for a line $(y = ax + b)$.

The solution will be of the form $f(x) = ax + b$

First, let's calculate $a$:
\[
a = (29-15) * (23-10)^{-1} \mod 71
\]

To find $(23-10)^{-1} \mod 71$, we need to find $x$ such that $13x \equiv 1 \mod 71$

We can use the extended Euclidean algorithm to find this inverse:

\begin{align*}
71 &= 5 \times 13 + 6 \\
13 &= 2 \times 6 + 1 \\
6 &= 6 \times 1 + 0
\end{align*}

Working backwards:
\begin{align*}
1 &= 13 - 2 \times 6 \\
  &= 13 - 2 \times (71 - 5 \times 13) \\
  &= 11 \times 13 - 2 \times 71
\end{align*}

\[
a = 14 * 60 \mod 71 = 840 \mod 71 = 59
\]

Next, we can find $b$ by substituting $x=10$ and $y=15$ into $y = ax + b$:
\[
15 \equiv 59 * 10 + b \mod 71
15 \equiv 590 + b \mod 71
15 \equiv 22 + b \mod 71
b \equiv 15 - 22 = 15 + 49 = 64 \mod 71
\]

Therefore, the polynomial is:
\[
f(x) = 59x + 64 \mod 71
\]

we check this is indeed correct
\begin{itemize}
    \item $f(10) = 59 * 10 + 64 \mod 71 = 590 + 64 \mod 71 = 15 \mod 71$
    \item $f(23) = 59 * 23 + 64 \mod 71 = 1 \neq 29 \mod 71$
\end{itemize}

This confirms that $f(x) = 56x + 25 \mod 71$ passes through the points $(10, 15)$ and $(23, 29)$ in the finite field with $p=71$.

\subsection*{Problem 8}
What is Lagrange interpolation and what does it do?

Lagrange interpolation is a method for finding the unique polynomial of degree $n-1$ that passes thorugh all $n$ points.

Find a polynomial that crosses through the points $(0, 1)$, $(1, 2)$, $(2, 1)$.

\begin{lstlisting}
    import galois

    GF = galois.GF(71)

    x = GF([0, 1, 2])
    y = GF([1, 2, 1])

    f = galois.lagrange_poly(x, y)

    print(f"Lagrange polynomial: {f}")

    # Verify the polynomial passes through the given points
    for xi, yi in zip(x, y):
        if not f(xi) == yi:
            assert False, f"f({xi}) = {f(xi)} (expected {yi})"
\end{lstlisting}

the answer is: $70x^2 + 2x + 1$

Use this Stack Overflow answer as a starting point: \url{https://stackoverflow.com/a/73434775}



\end{document}